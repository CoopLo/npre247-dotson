\documentclass{article}

\usepackage{amsfonts}
\usepackage{hyperref}
\usepackage{graphicx}
\usepackage{amsmath}
\usepackage{amssymb}

\author{Sam Dotson}
\title{NPRE 247 - Exam 2 Study Guide}

\begin{document}
\maketitle

\section{Binary Nuclear Reactions}
\begin{enumerate}
	\item{Connect concepts of particle collisions and decay to binary reactions}\\
	\item{Categorize nuclear reactions using standard nomenclature}\\
	\item{apply conservation of nucleaons to binary nuclear reactions.}\\
	\item{Formulate Q value equations for binary nuclear reactions}\\
	\item{Apply conservation of energy and linear momentum to scattering.}\\
	\item{Apply coulombic threshold.}\\
	\item{Apply kinematic threshold.}\\
	\item{Determine when coulombic and kinematic thresholds apply or do not.}\\
\end{enumerate}

\section{Radiation Interactions with Matter}

\begin{enumerate}
	\item{Define uncollided flux}\\
	\item{Define lnear interaction coefficient}\\
	\item{Apply linear interaction coefficients to a slab problem}\\
	\item{Identify the units of intensity, flux density, fluence, reaction rate}\\
	\item{Compare linear interaction coefficient and cross section}\\
	\item{Calculate uncollided flux in a medium}\\
	\item{Calculate mean free path of a particle in a medium}\\
	\item{Define the half thickness in a medium}\\
	\item{Apply the concept of buildup factor to attenuation in a slab}\\
	\item{Define microscopic cross section}\\
	\item{Calculate macroscopic cross sections, given a microscopic cross section}\\
	\item{Calculate the mass interaction coefficients of mixtures}\\
	\item{Calculate flux density}\\
	\item{Calculate Reaction Rate Density}\\
	\item{Recognize the dependence of flux on energy, position, and time}\\
	\item{Define radiation fluence}\\
	\item{Calculate uncollided flux density from isotropic point sources}\\
	\item{Apply the Kelin-Nishina formula to Compton Scattering}\\
	\item{Compare energy dependence of photon interaction cross sections}\\
	\item{Describe energy dependence of neutron interaction cross sections}\\
	\item{Recognize the comparative range of heavy vs. light particles}\\
	\item{Recognize the comparative range of charged particles}\\
\end{enumerate}

\section{Detection and Measurement of Radiation}
$\textit{This section is particularly relevant to my work in isotope identification.}$\\

\begin{enumerate}
	\item{}
	\item{}
	\item{}
	\item{}
	\item{}
	\item{}
	\item{}
	\item{}
	\item{}
	\item{}
\end{enumerate}

\section{Radiation Dosimetry}

\section{Neutron Diffusion and Moderation}
\textit{For this section, read Chapter 5 in Lamarsh textbook}\\\\
\textit{Also, read section 10.1 in Shultis for Neutron Moderation.}\\\\

\section{Criticality}

\section{6 Factor Formula}
$\textit{This section involves the neutron life cycle in a reactor.}$\\
$\textit{See section 10.4 in Shultis}$\\
$\textit{See chapter 6 in Lamarsh textbook}$\\

\begin{enumerate}
	\item{Fast fission factor $\epsilon$}\\
	\item{Resonance escape probability $\textit{p}$}\\
	\item{Thermal utilization $\textit{f}$}\\
	\item{Thermal fission factor $\eta$}\\
	\item{Thermal non-leakage probability $P_{\textit{NL}}^{\texit{th}}$}\\
	\item{Fast non-leakage probability $P_{\textit{NL}}^{\texit{f}}$}\\
	\item{Describe the complete life cycle of neutrons in a reactor.}\\
	\item{Effective multiplication factor $\textit{$k_{eff}$}$}
\end{enumerate}




\end{document}