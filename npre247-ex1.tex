\documentclass{article}

\usepackage{amsfonts}
\usepackage{hyperref}

\author{Sam Dotson}
\title{NPRE 247 - Exam 1 Study Guide}

\begin{document}
\maketitle

This document is a short summary of main topic listed under "learning objectives" for Prof. Katy Huff's 
\href{https://github.com/katyhuff/npre247}{NPRE 247 class}.

\section{Fundamentals}

\begin{enumerate}
	\item{Physical units important to nuclear engineering}
		\begin{enumerate}
			\item{The electron volt, eV (particularly MeV)}
			\item{Atomic mass units, $\mathcal{A}$}
			\item{}
		\end{enumerate}
	\item{Derive the electron volt}
		The electron volt is defined as the amount of work required to move a charge of 1$e^{-}$ across a potential of one volt. \\
		A volt has units of joules per coulomb (J/C), and an electron has a charge equal to $-1.602*10^{-19}$C so, 
		\begin{equation}
			e * V  = J,
		\end{equation}
		where J is the amount of energy/work required to move an electron across a 1 volt potential. 
	\item{Define the atomic mass unit}
		An atomic mass unit is equal to roughly 1/12 the mass of a Carbon-12 atom. It is also equivalent to the mass in grams of one mole of a particle, i.e. 1 mole of protons is approximately 1 gram. 1 mole of Carbon-12 is approximately 12 grams. 
	\item{Place neutrons, protons, and electrons within the Standard Model}
		Neutrons and protons and electrons are all fermions. They cannot exist in the same state and be in the same physical space. This is the pauli exclusion principle. \\\\
		Photons are an example of a boson, or force carrier. They carry electromagnetic forces.
	\item{Explain the significance of the A/Z X nomenclature}
		A = The mass number of an atom ($m_{n}$ + $m_{p}$\\
		Z = The number of protons of an atom\\
		X = The atomic symbol of an atom (Carbon --> C).
	\item{Define nuclide, isotope, isobar, isomer, isotone}
		Isotope = Identical numbers of protons, different numbers of neutrons. Same atom different mass numbers.\\
		Isomer = Same mass number and protons, different excitation. \\
		Isotone = Same number of neutrons, different number of protons.\\
		Isobar = Same number of nucleons (mass number). \\
		Nuclide = A specific A, Z, X combination.
	\item{Calculate relative atomic masses from $\gamma_{i}$ atomic abundances}
		The relative atomic mass, $\mathcal{A}$ can be calculated by, 
		\begin{equation}
			\mathcal{A}_X * 100/\gamma = \mathcal{A}_i , 
		\end{equation}
		Where $\mathcal{A}_i$ is the i-th isotope of an element.
	\item{Calculate atom density, isotope num. density, and molecule density from combinations of $\rho$, $\mathcal{A}$, $\gamma_{i}$, $N_{A}$, $w_{i}$}
	\item{Estimate the diameter of an atom}
	\item{Identify major sources of energy worldwide and domestically}
	\item{Differentiate between energy and electricity usage.}
	\item{Distinguish among the major modes of energy production.}
	\item{Map the geography of nuclear power production.}
\end{enumerate}

\section{Modern Physics}
\begin{enumerate}	
	\item{State the postulates of Special Relativity}
	\item{Differentiate between inertial/non-inertial frames}
	\item{Compare classical and relativistic physics approaches}
	\item{Calculate relativistic masses}
	\item{Calculate time dilation}
	\item{Calculate length contraction}
	\item{Describe the photoelectric effect}
	\item{Calculate compton scattering angles and kinetic energies}
	\item{Calculate deBroglie wavelengths}
	\item{Relate energy and wavelength for photons}
	\item{Understand wave-particle duality}
	\item{Relate wave-particle duality and particle energy/wavelength}
\end{enumerate}

\section{Nuclear Models}

\begin{enumerate}
	\item{Summarize the history of atomic theory development}
	\item{Recognize the radiation signatures that drove atomic theory}
	\item{List and describe the atomic models}
	\item{List and describe the nuclear models}
	\item{Explain the structure of the chart of the nuclides}
\end{enumerate}

\section{Energetics}

\begin{enumerate}		
	\item{Define exothermic and endothermic nuclear reactions}
	\item{Calculate binding energies}
	\item{Notation for binary reactions}
	\item{Physics of common binary reactions}
	\item{Explain the relationship between Q-value, mass, and energy in a reaction}
	\item{Energy and charge conservation in binary reactions}
	\item{Calculate Q-values for various reactions} 
\end{enumerate}	

\section{Radioactivity}

\begin{enumerate}
	\item{Explain the discovery of radioactivity}
	\item{What causes radioactive decay}
	\item{Sources of manmade and natural radiation}
	\item{Read and understand a decay diagram}
	\item{Describe the various types of decay}
	\item{Derive the radioactive decay law. Also write it in terms of half-lives.}
	\item{Calculate simple decay with production} 
\end{enumerate}	

\end{document}
